%%%
%%% v0.1: sent to PCP
%%%

\documentclass[a4paper]{moderncv}
%%%%%%%%%%%%%%%%%%%%%%%%%%%%%%%%%%%%%%%%%%%%%%%%%%%%%%%%%%%%%%%%%%%%%%%
%%% specifying options for CV
%%%%%%%%%%%%%%%%%%%%%%%%%%%%%%%%%%%%%%%%%%%%%%%%%%%%%%%%%%%%%%%%%%%%%%%
\newboolean{cvit}
\newboolean{cvaccademic}
\newboolean{cvfull}
\newboolean{cvpoli}

\setboolean{cvit}{false}
\setboolean{cvaccademic}{false}
\setboolean{cvfull}{false}
\setboolean{cvpoli}{false}

%\extendedversion{La versione estesa del CV \`e disponibile a \url{www.sli-m.com/vernizzi.pdf}}

%%%%%%%%%%%%%%%%%%%%%%%%%%%%%%%%%%%%%%%%%%%%%%%%%%%%%%%%%%%%%%%%%%%%%%%
%%% Misc includes and commands
%%%%%%%%%%%%%%%%%%%%%%%%%%%%%%%%%%%%%%%%%%%%%%%%%%%%%%%%%%%%%%%%%%%%%%%
\usepackage[utf8]{inputenc}
%\usepackage{hyperref}
\usepackage{graphicx}
\usepackage{float}

\moderncvtheme[blue]{classic}

\newcommand{\structure}[1]{\color{sectiontitlecolor}\textbf{#1}\color{black}}

\newcommand{\summary}{Summary}
\newcommand{\education}{Education}
\newcommand{\experience}{Experience}
\newcommand{\teaching}{Teaching}
\newcommand{\research}{Scientific Activity}
\newcommand{\projects}{Research Projects}
\newcommand{\languages}{Languages}
\newcommand{\skills}{Computer Skills}
\newcommand{\interests}{Interests and Activities}
\newcommand{\publications}{Publications (Selection)}
\newcommand{\misc}{Other Information}

\newcommand{\italian}[2]{\ifthenelse{\boolean{cvit}}{#1}{#2}}
\newcommand{\accademic}[1]{\ifthenelse{\boolean{cvaccademic}}{#1}{}}
\newcommand{\work}[1]{\ifthenelse{\boolean{cvaccademic}}{}{#1}}
\newcommand{\full}[1]{\ifthenelse{\boolean{cvfull}}{#1}{}}
\newcommand{\short}[1]{\ifthenelse{\boolean{cvfull}}{}{#1}}
\newcommand{\poli}[2]{\ifthenelse{\boolean{cvpoli}}{#1}{#2}}

\newcommand{\otc}{Open\_TC}

%%%%%%%%%%%%%%%%%%%%%%%%%%%%%%%%%%%%%%%%%%%%%%%%%%%%%%%%%%%%%%%%%%%%%%%
%%% Settings for italian language
%%%%%%%%%%%%%%%%%%%%%%%%%%%%%%%%%%%%%%%%%%%%%%%%%%%%%%%%%%%%%%%%%%%%%%%
\italian{
	\usepackage[italian]{babel}

	\renewcommand{\summary}{Competenze ed interessi}
	\renewcommand{\education}{Formazione}
	\renewcommand{\experience}{Esperienze di lavoro}
	\renewcommand{\teaching}{Attivit\`a didattica}
	\renewcommand{\research}{Attivit\`a scientifica}
	\renewcommand{\projects}{Partecipazione a progetti di ricerca}
	\renewcommand{\languages}{Lingue}
	\renewcommand{\skills}{Competenze informatiche}
	\renewcommand{\interests}{Interessi e attivit\`a}
	\renewcommand{\refname}{Pubblicazioni (Selezione)}
	\renewcommand{\misc}{Altre informazioni}
}
{
	\renewcommand{\summary}{Summary}
	\renewcommand{\education}{Education}
	\renewcommand{\experience}{Work experience}
	\renewcommand{\teaching}{Attivit\`a didattica}
	\renewcommand{\research}{Attivit\`a scientifica}
	\renewcommand{\projects}{Research projects}
	\renewcommand{\languages}{Languages}
	\renewcommand{\skills}{Skills}
	\renewcommand{\interests}{Interessi e attivit\`a}
	\renewcommand{\refname}{Pubblicazioni (Selezione)}
	\renewcommand{\misc}{Languages and other information}
}

%%%%%%%%%%%%%%%%%%%%%%%%%%%%%%%%%%%%%%%%%%%%%%%%%%%%%%%%%%%%%%%%%%%%%%%
%%% Settings for work CV
%%%%%%%%%%%%%%%%%%%%%%%%%%%%%%%%%%%%%%%%%%%%%%%%%%%%%%%%%%%%%%%%%%%%%%%
\work{
	\nopagenumbers
}

%%%%%%%%%%%%%%%%%%%%%%%%%%%%%%%%%%%%%%%%%%%%%%%%%%%%%%%%%%%%%%%%%%%%%%%
%%% PERSONAL DATA
%%%%%%%%%%%%%%%%%%%%%%%%%%%%%%%%%%%%%%%%%%%%%%%%%%%%%%%%%%%%%%%%%%%%%%%
\firstname{Davide}
\familyname{Vernizzi}
\title{Curriculum Vit\ae}
\address{Via Belfiore, 42}{10125, Turin, Italy}
\mobile{+39 329 24 73 484}
\email{davide.vernizzi@gmail.com}
%\homepage{\url{www.sli-m.com}}

\begin{document}
\maketitle

%%%%%%%%%%%%%%%%%%%%%%%%%%%%%%%%%%%%%%%%%%%%%%%%%%%%%%%%%%%%%%%%%%%%%%%
%%% Competenze ed interessi
%%%%%%%%%%%%%%%%%%%%%%%%%%%%%%%%%%%%%%%%%%%%%%%%%%%%%%%%%%%%%%%%%%%%%%%
\work {
%	\vspace{-3.5em}  %%% Hack to fit in 2 pages

        I am currently working as a software engineer at Ennova Srl,
        which was awarded with the prize of Italian Startup of Year 2014.

        I develop web applications, backends, and APIs for mobile applications,
        which are mainly used by telecommunication companies and energy providers,
        such as Telecom Italia, Vodafone, Enel, and Iren Energia.

        While working at Ennova, I have been in charge of integrating clients' databases,
        I developed web services used by clients’ IVR,
        and I coordinated the efforts of Ennova's developers in order to develop various mobile apps.
        Moreover, I personally developed some of the most critical server side building blocks in many projects,
        and I have helped to design Ennova's cloud computing architecture.

        Before working on web and mobile applications, I obtained a Ph.D in security of computer systems.
        In this context, I have participated in the European research project \otc,
        aimed at creating an open-source framework for Trusted Computing.
        Following this, I worked on security of cloud computing, contributing to the research project TClouds,
        whose goal was to create a resilient and privacy-friendly cloud computing platform.

        While holding my positions, both in university and in industry,
        I have always been involved into the training of students or new employees.
} %work

%%%%%%%%%%%%%%%%%%%%%%%%%%%%%%%%%%%%%%%%%%%%%%%%%%%%%%%%%%%%%%%%%%%%%%%
%%% EDUCATION
%%%%%%%%%%%%%%%%%%%%%%%%%%%%%%%%%%%%%%%%%%%%%%%%%%%%%%%%%%%%%%%%%%%%%%%
\section{\education}
\work {
	%%% Dottorato - 15/04/2010
	\cventry{15 April 2010}{Ph.D. in  Computer Science}{Politecnico di Torino}{}{}{%
		Thesis: On Trusted and Privacy-Friendly Network Communications.\newline{}
		Advisor: prof. Antonio Lioy.
	}

	%%% Laurea Poli - 04/05/2006
	\cventry{5 May 2006}{M.Sc. in Computer Engineering}{Politecnico di Torino}{}{}{%{\textit{104/110}}{%
		Thesis: Self-adaptive parallel algorithms for computer vision applications.
		This thesis was written in collaboration with the \'Ecole Nationale Sup\'erieure d'Informatique et des Math\'ematiques Appliqu\'ees de Grenoble (ENSIMAG).\newline{}
		Advisors: prof. Bartrolomeo Montrucchio (POLITO), prof. Jean-Louis Roch (ENSIMAG).
	}

	%%% Laurea ENSIMAG - 11/07/2005
	\cventry{11 July 2005}{Dipl\^ome d'Ing\'enieur (equivalent to M.Sc. in Computer Engineering)}{\'Ecole Nationale Sup\'erieure d'Informatique et des Math\'ematiques Appliqu\'ees de Grenoble (ENSIMAG)}{}{}{%
		Thesis: Self-adaptive parallel algorithms for computer vision applications.\newline{}
		Advisor: prof. Jean-Louis Roch (ENSIMAG).
	}

	% Laurea breve - 12/09/2003
	%\cventry{12 Sept. 2003}{B.Sc. in Computer Engineering}{Politecnico di Torino}{}{}{%{\textit{104/110}}{% }
} %work

%%%%%%%%%%%%%%%%%%%%%%%%%%%%%%%%%%%%%%%%%%%%%%%%%%%%%%%%%%%%%%%%%%%%%%%
%%% WORK EXPERIENCE
%%%%%%%%%%%%%%%%%%%%%%%%%%%%%%%%%%%%%%%%%%%%%%%%%%%%%%%%%%%%%%%%%%%%%%%
\section{\experience}

\work {
    % Inizio: 01/11/2011
		\cventry{\textbf{since 2011}}{Developer}{Ennova s.r.l, \url{http://www.ennova.it/}}{}{}{%
            Development of web and mobile applications for customer care.
            Integration and development of enterprise services on behalf of telephone companies and energy providers.
		}

        \cventry{\textbf{2010 -- 2011}}{Postdoctoral researcher}{Politecnico di Torino, Dip. di Automatica e Informatica}{}{}{%
            Following the Ph.D I continued my research activity at Politecnico di Torino, focusing on \emph{Cloud Computing}.
        }

		\cventry{\textbf{2006 -- 2010}}{Research activity}{Politecnico di Torino, Dip. di Automatica e Informatica}{}{}{%
            Research activity at Politecnico di Torino on security of computer systems and network.
            The research activities were mainly focused to \emph{Trusted Computing} and its application
            to network of computers, with a particular attention to privacy.
        }
		
		\cventry{\textbf{2009--2010}}{Consultant}{SmartRM S.r.l., \url{http://www.smartrm.com}~}{Torino}{}{%
            Consulting activity for SmartRM, a company specialized in developing encryption software
            that allows users to easily protect and share digital content.
            I helped SmartRM to integrate \emph{Trusted Computing} principles to content protection.
        }

		\cventry{\textbf{since 2003}}{Consultant}{}{}{}{%
            I did consulting for many different projects.
        }

		%\cventry{\textbf{2005}}{Internship}{ID-IMAG, \url{http://bit.ly/id-imag}}{Grenoble, Francia}{}{%
        %    Study and development of parallel algorithms used for computer vision.
        %}

		% Inizio XX/06/2004
		% Inizio XX/05/2003
		%\cventry{\textbf{2003, 2004}}{Internship}{Dynamic Fun S.r.l., \url{http://www.dynamicfun.com/}~}{Torino}{}{%
        %    Study and development of algorithms for face detection
        %}
} %work
% work


\work{
	\section{\projects}

	%%%%%%%%%%%%%%%%%%%%%%%%%%%%%%%%%%%%%%%%%%%%%%%%%%%%%%%%%%%%%%%%%%%%%%%
	%%% Descrione dei vari progetti di ricerca
	%%%%%%%%%%%%%%%%%%%%%%%%%%%%%%%%%%%%%%%%%%%%%%%%%%%%%%%%%%%%%%%%%%%%%%%
    \cventry{\textbf{2010-2011}}{TClouds (Trustworthy Clouds)}{}{}{\url{www.tclouds-project.eu} }{%
        TClouds is EU-funded research project. Its main goal is to develop an advanced cloud infrastructure
        that can deliver computing and storage that achieves a new level of security, privacy, and resilience.
        \newline{}
%        \full{EC - DG/INFSO - FP7 (ICT-2009-257243).\newline{}}
        \work{Main partners: IBM, Philips, University of Oxford.} 
    }

        \cventry{\textbf{2010-2011}}{STORK (Secure Identity Across Borders Linked)}{}{}{\url{www.eid-stork.eu} }{%
            STORK project was funded by the European Commission as part of its Competitiveness and Innovation Programme. Its aim is to establish a European electronic ID Interoperability Platform that will allow citizens to use their national eID for authenticating and accessing electronic services across Europe.
            \newline{}
            \work{Main partners: 14 EU Member States.}
        }

%        \cventry{\textbf{2009--2010}}{Trusted Platform Agent}{\url{http://goo.gl/mLodg4} }{}{}{%
%		C library (Linux, Windows) for application exploiting Trusted Computing and TPM.
%		}

        \cventry{\textbf{2007--2009}}{Open\_TC (Open Trusted Computing)}{}{}{\url{http://goo.gl/IDlKeB} }{%
            Development of a framework for Trusted Computing based on open source software.
            \newline{}
            \work{Main partners: IBM, HP, Infineon, AMD.}
        }

%%%%%%%%%%%%%%%%%%%%%%%%%%%%%%%%%%%%%%%%%%%%%%%%%%%%%%%%%%%%%%%%%%%%%%%
%%% TECH SKILLS
%%%%%%%%%%%%%%%%%%%%%%%%%%%%%%%%%%%%%%%%%%%%%%%%%%%%%%%%%%%%%%%%%%%%%%%
\section{\skills}

		\cvcomputer{OS}{Linux, Mac OS X, Windows}{Altro}{MySQL, UML, Vim, \LaTeX, Office}
		\cvcomputer{Programming}{C, Java, Python}{Web}{PHP, Javascript, HTML, CSS}	

		%\cventry{Programming library}{DAA Toolkit}{\url{http://security.polito.it/tc/daa} }{}{}{%
		%Patch for OpenSSL for anonymous authentication using TLS and DAA.
		%}

		%\cventry{Web}{sli-m.com: The \structure{sli}de \structure{m}achine}{\url{http://sli-m.com}}{}{}{%
		%Web service for making presentation basing on mind maps and \LaTeX/Beamer.
		%}
} % work


	%%%%%%%%%%%%%%%%%%%%%%%%%%%%%%%%%%%%%%%%%%%%%%%%%%%%%%%%%%%%%%%%%%%%%%%
	\section{\misc}
	%%%%%%%%%%%%%%%%%%%%%%%%%%%%%%%%%%%%%%%%%%%%%%%%%%%%%%%%%%%%%%%%%%%%%%%
		\cvline{\textbf{Languages}}{Native language: Italian. Fluent in English and French.}
		\cvline{\textbf{Teaching}}{Teaching at Politecnico di Torino as teacher assistant and teaching to private companies as teacher.}



%%%%%%%%%%%%%%%%%%%%%%%%%%%%%%%%%%%%%%%%%%%%%%%%%%%%%%%%%%%%%%%%%%%%%%%
\makeatletter
\renewcommand*{\bibliographyitemlabel}{\@biblabel{\arabic{enumiv}}}
\makeatother

\accademic{
\ \\\ \\
Torino, \today
\flushright
Davide VERNIZZI\hspace*{1cm}
} % accademic

\end{document}

